\documentclass[12pt]{report}
 
\usepackage[utf8]{inputenc}
\usepackage[T1]{fontenc}
\usepackage[francais]{babel}
\usepackage[top=2cm, bottom=2cm, left=2cm, right=2cm]{geometry}
\usepackage{amsmath}
\usepackage{amssymb}
\usepackage{mathrsfs}
\usepackage{color} 
\usepackage{colortbl}
\usepackage{slashbox}          
                           
\title{FISSURATION D'ELEMENTS EN BETON ET EFFETS D'ECHELLE}
\author{\textsc{Christian Nader}\\Encadrement: \textsc{C. Olivier-Leblond,C. Giry, F. Ragueneau}}
\date{\\Le 20 Juin 2013} 
  
\begin{document}
  
\maketitle

\chapter*{Remerciements}
thank you thank you've been a great audience

\renewcommand{\contentsname}{Sommaire} 
\tableofcontents 

\chapter*{Introduction}
\addcontentsline{toc}{chapter}{Introduction} 
Dans ce stage on propose un protocole expérimental permettant de mettre en œuvre des essais de
flexion sur poutre en béton mettant en évidence des effets d’échelle en termes de fissuration.
Les résultats de l’étude faite sur des poutres en micro-béton (à priori) entaillées et non entaillées
chargées en flexion trois points seront comparés à d’autres résultats expérimentaux et les lois
d'effet d’échelle de Bažant. On confirme déjà l’existence de l’effet d’échelle sur des poutres de
petites tailles (ce qui est le cas de notre étude). Cependant, pour de plus grandes poutres non
entaillées la résistance nominale tend vers une valeur constante liée à la résistance en traction
uniaxiale. Il a été conclu que l'applicabilité de la loi d’effet d’échelle dépend du type de
problème, ceci dit, si la propagation de la fissure avant la charge ultime est très stable, la loi
d’effet d’échelle peut être utilisée dans une gamme de taille plutôt large. Toutefois, si ce n'est pas
le cas, la validité de la loi d’effet d’échelle est limitée à une plage de taille plus petite. Par
conséquent, on ne peut pas extrapoler l'effet d’échelle à partir des essais avec une gamme de
taille petite à une gamme de grande taille, pour cela il nous manque des données expérimentales
provenant d’essais sur de gammes de grande taille. Ainsi, dans notre étude, nous allons ignorer
cette partie de la problématique, puisque nous allons tester seulement de petites poutres, plutôt,
nous allons nous concentrer sur d'autres aspects de l’effet d’échelle tels que les propriétés de la
résistance nominale d’une structure contenant une entaille (ou initialement une grande fissure) et
d’autres initialement non entaillées.


Mots-clés: Flexion trois points, Béton, Fissure, Effet d’échelle


\part{Bibliographie}
\chapter{Effet d'échelle}
\section{Introduction}


\section{Loi en puissance}


\section{Analyse asymptotique dans les poutres entaillées}


\section{Effet d'échelle à l'amorcage de la fissure}


\section{Effet d'echelle en flexion trois points}


\chapter{Energie de fissuration}
\section{Introduction}


\section{Méthode de détermination}



\chapter{correlation d'image}
\section{Introduction}


\section{Principe}


\section{Discrétisation par éléments finis (Correli Q4)}


\part{Procédure expérimentale}
\chapter*{}
Cette section est consacrée à la présentation de la campagne expérimentale menée sur des
poutres homothétiques, entaillées et non-entaillées faites du même matériau. Nous avons
considéré trois géométries différentes afin de prendre en compte les effets d’échelle et les effets
de bord.

\chapter{Les échantillons}
\section{Description du matériau}
Dans notre étude on test des éprouvettes de petites dimensions on a alors recours à utiliser un
micro-béton. (Référence: rapport de stage de Amine Hamouche - 2010)
\\\\
La composition du micro-béton pour $1m^3$ est présenté dans le tableau 1. Le ciment utilisé
est un CEM1 $52,5 N$ (HOLCIM), et le sable normalisé est conforme $ISO  679$, sa granulométrie
est $0/2 mm$ et donnée dans le tableau 2. Le rapport $e/c$ est de $0,46$. Cette formulation est tirée de
la thèse de Anna Ouglova 2004.

\begin{table}[h]
\begin{center}
\begin{tabular}{ccc}
\hline
\begin{bf}                   Sable(Kg)         \end{bf} & \begin{bf}                   Eau(Kg)         \end{bf} & \begin{bf}                   Ciment(Kg)           \end{bf} \\
\hline 
\begin{bf}\rowcolor{cyan}1342\end{bf} & \begin{bf}294\end{bf} & \begin{bf}631\end{bf} \\
\hline 
\end{tabular}
\end{center}
\caption{Composition du micro-béton}
\label{micro-béton}
\end{table}

\begin{table}[h]
\begin{center}
\begin{tabular}{cc}
\hline
\begin{bf}Tamis ouverture des mailles (mm)\end{bf}   &   \begin{bf}refus cummulés (\%)\end{bf} \\
\hline 
\begin{bf}\rowcolor{cyan}0.08\end{bf}   &   \begin{bf}99 \pm{}1 \end{bf} \\
 
\begin{bf}0.16\end{bf}   &   \begin{bf}87 \pm{}5\end{bf} \\
 
\begin{bf}\rowcolor{cyan}0.50\end{bf}   &   \begin{bf}67 \pm{}5\end{bf} \\

\begin{bf}1.00\end{bf}   &   \begin{bf}33 \pm{}5\end{bf} \\

\begin{bf}\rowcolor{cyan}1.60\end{bf}   &   \begin{bf}7 \pm{}5\end{bf} \\

\begin{bf}2.00\end{bf}   &   \begin{bf}0\end{bf} \\
\hline 
\end{tabular}
\end{center}
\caption{Granulométrie du sable normalisé}
\label{sable}
\end{table}


\section{Géométrie des éprouvettes}
Les essais de flexion seront réalisés sur des poutres de dimensions $B\times D\times L{} cm^3 (largeur \times{} hauteur
\times{} longueur)$ avec un rapport $L/D=5$ constant et $B$ une dimension fixe (restriction du type de test,
ici $B=4 cm$).
\\\\
Trois différentes tailles sont considérées, présentant toutes un rapport $L/D = 5$, où $L$ est la
longueur et $D$ est la hauteur de la poutre. La profondeur, quant à elle a été choisie constante et
égale à $B (4 cm)$. Les entailles centrales seront taillées après coulage et non moulées sur les
éprouvettes. Trois longueurs d’entaille ont été considérées avec une épaisseur d’entaille
constante de $2 mm$ pour tous les échantillons (voir figure 4, page suivante).
\\\\
On propose les dimensions suivantes :
\begin{table}[h]
\begin{center}
\begin{tabular}{c|c|c|c|c}
\hline
\backslashbox{\begin{bf}taille\end{bf}}{\begin{bf}dimension\end{bf}}   &   \begin{bf}épaisseur\end{bf}   &   \begin{bf}hauteur\end{bf}   &   \begin{bf}longueur\end{bf}   &   \begin{bf}distance entre appuis\end{bf}\\
\hline
 
\begin{bf}\rowcolor{cyan}$n_1$\end{bf}   &   \begin{bf}$B_1=4cm$\end{bf}   &   \begin{bf}$D_1=4cm$\end{bf}   &   \begin{bf}$L_1=20cm$\end{bf}   &   \begin{bf}$I_1=16cm$\end{bf} \\

\begin{bf}$n_2$\end{bf}   &   \begin{bf}$B_2=4cm$\end{bf}   &   \begin{bf}$D_2=8cm$\end{bf}   &   \begin{bf}$L_2=40cm$\end{bf}   &   \begin{bf}$I_2=32cm$\end{bf} \\

\begin{bf}\rowcolor{cyan}$n_3$\end{bf}   &   \begin{bf}$B_3=4cm$\end{bf}   &   \begin{bf}$D_3=8cm$\end{bf}   &   \begin{bf}$L_3=80cm$\end{bf}   &   \begin{bf}$I_3=64cm$\end{bf} \\

\hline 
\end{tabular}
\end{center}
\caption{Les dimensions des éprouvettes}
\label{sable}
\end{table}

%%%%%%%%%%%%%%%%%%%%%%%%%%%%%%%%%%%%%%%%%%%%%%%%%%%%%


\section{Coffrages}
Types de bois :\\
\begin{itemize}
 
\item[•] bois standard (BS, joue le rôle de support)
\item[•] bois de coffrage (BC, en contact avec le béton)\\

\end{itemize}
Pour concevoir le coffrage on utilise des planches de bois standard de $2,5m \times{} 1,25m$
(épaisseur $1,5 cm$), ainsi que des planches de bois de coffrage de $2,5m \times{} 1,22m$ (épaisseur $1,8 cm$), on admet $4 mm$ comme trait de coupe.
\\\\
Les coffres seront conçus d’une façon à être réutilisables, on aura besoin alors de tiges
filetés ($2$ par éprouvette).
\\\\
On a besoin de $9$ coffres en total mais les coffres des éprouvettes qui ont une même09088
dimension seront exactement pareils.
\\
\begin{itemize}
 
\item[•] Pour les éprouvettes de dimension $4\times 4\times 20$ on peut faire un seul coffre de la manière
suivante : (N.B. : masse du coffre avec béton environ $6,5Kg$)


\end{itemize}

\section{Matériel et équipement}
Composants du micro-béton (perte 15\%) :\\
Volume total de béton :\\
\begin{itemize} 
\item[\triangleright] $27$ éprouvettes (poutres) : $v_1 = 60480 cm^3$
\item[\triangleright] $12$ échantillons cylindriques ($11\times 22$) pour définir les caractéristiques du béton (6 pour le
   test de compression uni-axial, 6 pour l’essai brésilien) : $v_2 = 25090 cm^3$   
\item[\triangleright] $+5L$ (cône d’Abrams) $+1L$ (Aréomètre) : $v_3 = 6000 cm^3$
\item[\triangleright] Volume total : $V = 91570 cm^3$
\item[\triangleright] $+pertes$ : $V_{final} = 0.1053055 m^3$\\
\end{itemize}

Ainsi,d’après le Tableau -1- on peut calculer les quantités des composantes du béton:\\

\begin{itemize}
\item[•] Sable : $141.32 Kg$
\item[•] Eau : $30.96 Kg$
\item[•] Eau : Ciment : $66.45 Kg$\\
\end{itemize}

Matériels pour le coffrage :\\

\begin{itemize}
\item[•] Une plaque de bois standard
\item[•] Une plaque de bois de coffrage
\item[•] 16 écrous
\item[•] 16 rondelles
\end{itemize}

\chapter{Essai F3P}
\section{MTS}

\section{Pilotage}



\part{Étude numérique}
\chapter*{}
Dans notre étude on utilise comme matériau un micro-béton. Pour bien prédire numériquement
les résultats de l’expérience, on a besoin d’introduire les paramètres qui correspondes à ce
matériau, or par manque de données sur certains de ces paramètres (principalement la fragilité en
traction) on a recours à une identification par rapport à des essais expérimentaux. Pour cela on a
lancé plusieurs calcules sur des échantillons similaires aux échantillons de l’expérience en
géométrie et chargement et on a changé les paramètres d’une façon intuitive pour obtenir au final
une courbe force/déplacement similaire aux courbes trouvées expérimentalement.

\chapter{Modélisation}
\section{Mode de calcule}


\section{Paramètres matériau}


\section{Maillages}


\section{Résultats}


\part{Résultats et analyse expérimentale}
\chapter*{}


\chapter{Identification des paramètres mécanique}
\section{Procédure}


\section{Mesures}


\chapter{Essais de flexion trois points}
\section{montage}


\section{Résultats}


\chapter*{Références}
\addcontentsline{toc}{chapter}{Références} 
trallalla

\end{document}