\documentclass[12pt]{report}
 
\usepackage[latin1]{inputenc}
\usepackage[T1]{fontenc}
\usepackage[francais]{babel}
\usepackage[top=2cm, bottom=2cm, left=2cm, right=2cm]{geometry}
                           
\title{FISSURATION D'ELEMENTS EN BETON ET EFFETS D'ECHELLE}
\author{\textsc{Nader} \textsc{Christian}}
\date{lol} 
  
\begin{document}
  
\maketitle % Page de garde.

\chapter*{Remerciements}
thank you thank you've been a great audience
 
\tableofcontents

\chapter*{Introduction}
Dans ce stage on propose un protocole expérimental permettant de mettre en œuvre des essais de
flexion sur poutre en béton mettant en évidence des effets d’échelle en termes de fissuration.
Les résultats de l’étude faite sur des poutres en micro-béton (à priori) entaillées et non entaillées
chargées en flexion trois points seront comparés à d’autres résultats expérimentaux et les lois
d'effet d’échelle de Bažant. On confirme déjà l’existence de l’effet d’échelle sur des poutres de
petites tailles (ce qui est le cas de notre étude). Cependant, pour de plus grandes poutres non
entaillées la résistance nominale tend vers une valeur constante liée à la résistance en traction
uniaxiale. Il a été conclu que l'applicabilité de la loi d’effet d’échelle dépend du type de
problème, ceci dit, si la propagation de la fissure avant la charge ultime est très stable, la loi
d’effet d’échelle peut être utilisée dans une gamme de taille plutôt large. Toutefois, si ce n'est pas
le cas, la validité de la loi d’effet d’échelle est limitée à une plage de taille plus petite. Par
conséquent, on ne peut pas extrapoler l'effet d’échelle à partir des essais avec une gamme de
taille petite à une gamme de grande taille, pour cela il nous manque des données expérimentales
provenant d’essais sur de gammes de grande taille. Ainsi, dans notre étude, nous allons ignorer
cette partie de la problématique, puisque nous allons tester seulement de petites poutres, plutôt,
nous allons nous concentrer sur d'autres aspects de l’effet d’échelle tels que les propriétés de la
résistance nominale d’une structure contenant une entaille (ou initialement une grande fissure) et
d’autres initialement non entaillées.


Mots-clés: Flexion trois points, Béton, Fissure, Effet d’échelle


\part{Étude Bibliographique}

\chapter{Effet d'échelle}

\section{Introduction}
L'effet d’échelle dans les matériaux quasi fragile comme le béton est un phénomène bien connu
et il existe un certain nombre d'études expérimentales et théoriques (Rusch et al, 1962; Leonhardt
et Walter, 1962; Kani, 1967; Bhal, 1968; Taylor, 1972; Walsh 1976: Walraven et Lehwalter
1990; Chana 1981; Reinhardt, 1981 a,b; Iguro et al, 1985; Hillerborg, 1989; E1igehausen et al.
1992) qui confirment l'existence de celui-ci. Il ya deux aspects de l'effet d’échelle : statistique et
déterministe. Basés sur la mécanique de la rupture. Dans le passé, l'effet d’échelle a été
principalement traitée du point de vue statistique (Weibull, 1939; Mihashi et Zaitscv, 1981;
Mihashi, 1983). Cependant, actuellement il ya suffisamment de preuves qui montent que la
raison principale de l'effet d’échelle réside dans la libération de l'énergie de déformation due à la
croissance de la rupture.

Selon Bažant (1984) l'effet d’échelle peut être approximativement décrit par la loi d’effet
d’échelle:

 
\end{document}