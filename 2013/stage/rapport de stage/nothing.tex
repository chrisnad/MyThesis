\documentclass[12pt]{report}
 
\usepackage[utf8]{inputenc}
\usepackage[T1]{fontenc}
\usepackage[francais]{babel}
\usepackage[top=2cm, bottom=2cm, left=2cm, right=2cm]{geometry}
\usepackage{amsmath}
\usepackage{amssymb}
\usepackage{mathrsfs}
                           
\title{FISSURATION D'ELEMENTS EN BETON ET EFFETS D'ECHELLE}
\author{\textsc{Christian Nader}\\Encadrement: \textsc{C. Olivier-Leblond,C. Giry, F. Ragueneau}}
\date{\\Le 20 Juin 2013} 
  
\begin{document}
  
\maketitle

\chapter*{Remerciements}
thank you thank you've been a great audience

\renewcommand{\contentsname}{Sommaire} 
\tableofcontents 

\chapter*{Introduction}
\addcontentsline{toc}{chapter}{Introduction} 
Dans ce stage on propose un protocole expérimental permettant de mettre en œuvre des essais de
flexion sur poutre en béton mettant en évidence des effets d’échelle en termes de fissuration.
Les résultats de l’étude faite sur des poutres en micro-béton (à priori) entaillées et non entaillées
chargées en flexion trois points seront comparés à d’autres résultats expérimentaux et les lois
d'effet d’échelle de Bažant. On confirme déjà l’existence de l’effet d’échelle sur des poutres de
petites tailles (ce qui est le cas de notre étude). Cependant, pour de plus grandes poutres non
entaillées la résistance nominale tend vers une valeur constante liée à la résistance en traction
uniaxiale. Il a été conclu que l'applicabilité de la loi d’effet d’échelle dépend du type de
problème, ceci dit, si la propagation de la fissure avant la charge ultime est très stable, la loi
d’effet d’échelle peut être utilisée dans une gamme de taille plutôt large. Toutefois, si ce n'est pas
le cas, la validité de la loi d’effet d’échelle est limitée à une plage de taille plus petite. Par
conséquent, on ne peut pas extrapoler l'effet d’échelle à partir des essais avec une gamme de
taille petite à une gamme de grande taille, pour cela il nous manque des données expérimentales
provenant d’essais sur de gammes de grande taille. Ainsi, dans notre étude, nous allons ignorer
cette partie de la problématique, puisque nous allons tester seulement de petites poutres, plutôt,
nous allons nous concentrer sur d'autres aspects de l’effet d’échelle tels que les propriétés de la
résistance nominale d’une structure contenant une entaille (ou initialement une grande fissure) et
d’autres initialement non entaillées.


Mots-clés: Flexion trois points, Béton, Fissure, Effet d’échelle


\part{Bibliographie}
\chapter{Effet d'échelle}
\section{Introduction}


\section{Loi en puissance}


\section{Analyse asymptotique dans les poutres entaillées}


\section{Effet d'échelle à l'amorcage de la fissure}


\section{Effet d'echelle en flexion trois points}


\chapter{Energie de fissuration}
\section{Introduction}


\section{Méthode de détermination}



\chapter{correlation d'image}
\section{Introduction}


\section{Principe}


\section{Discrétisation par éléments finis (Correli Q4)}


\part{Procédure expérimentale}
\chapter*{}
Cette section est consacrée à la présentation de la campagne expérimentale menée sur des
poutres homothétiques, entaillées et non-entaillées faites du même matériau. Nous avons
considéré trois géométries différentes afin de prendre en compte les effets d’échelle et les effets
de bord.

\chapter{Les échantillons}
\section{Description du matériau}


\section{Géométrie des éprouvettes}


\section{Coffrages}


\section{Matériel et équipement}


\chapter{Essai F3P}
\section{MTS}


\section{Pilotage}

\part{Étude numériques}
\chapter*{}
Dans notre étude on utilise comme matériau un micro-béton. Pour bien prédire numériquement
les résultats de l’expérience, on a besoin d’introduire les paramètres qui correspondes à ce
matériau, or par manque de données sur certains de ces paramètres (principalement la fragilité en
traction) on a recours à une identification par rapport à des essais expérimentaux. Pour cela on a
lancé plusieurs calcules sur des échantillons similaires aux échantillons de l’expérience en
géométrie et chargement et on a changé les paramètres d’une façon intuitive pour obtenir au final
une courbe force/déplacement similaire aux courbes trouvées expérimentalement.

\chapter{Modélisation}
\section{Mode de calcule}


\section{Paramètres matériau}


\section{Maillages}


\section{Résultats}


\part{Résultats et analyse expérimentale}
\chapter*{}


\chapter{Identification des paramètres mécaniques
}
\section{Procédure}


\section{Mesures}


\chapter{Essais de flexion trois points}
\section{montage}


\section{Résultats}


\chapter*{Références}
\addcontentsline{toc}{chapter}{Références} 
trallalla

\end{document}